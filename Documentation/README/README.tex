%% Incluindo os pacotes necessários
\documentclass[12pt, a4paper]{article}
\usepackage[utf8]{inputenc}
\usepackage[portuguese]{babel}
\usepackage{titlesec}
\usepackage{titling}
\usepackage{enumitem}
\usepackage{indentfirst}
\usepackage{graphicx}
%%\graphicspath{{images/}}
\usepackage{fancyhdr}
\usepackage{color}
\usepackage{fancyhdr}
\usepackage{colortbl}
\usepackage{framed}

%% Definindo o Autor e o título
\author{Gustavo Juliano Borges \and Victor Emanuel Almeida \and Marco Aurélio Guerra Pedroso}
\title{Documentação do Projeto I~-~IES}

%% Estilo da página
\pagestyle{fancy}
\fancyhead[L]{Documentação~-~Projeto I}
\fancyhead[R]{Introdução Engenharia de Software}
\fancyfoot[L]{Outubro~-~2020}
\fancyfoot[C]{}
\fancyfoot[R]{-~\thepage~-}
\renewcommand{\headrulewidth}{0.7pt}
\renewcommand{\footrulewidth}{0.3pt}

%% Definindo espaçamento
\titlespacing{\section}{0pt}{*2}{*1.1}
%%\titlespacing{\subsection}{}{}{}

\titleformat{\section}
{\Large\bfseries}
{\thesection}
{.5cm}
{}

\pretitle{\vfill\begin{center}\LARGE}
\postdate{\end{center}\vfill}

\begin{document}
\begin{titlepage}
\maketitle\thispagestyle{empty}
\end{titlepage}
\section{Introdução}

Foi proposto pelo professor a elaboração de um programa que simula um streaming de vídeo.

Sendo assim criamos uma equipe formada por~\textbf{\theauthor}, e nomeamos nosso software de \textbf{Data Streaming Video System} \textbf{(DSVS)}

\section{Casos de uso}
\subsection{Cadastrar um vídeo}
\begin{enumerate}
	\item Entrar no menu Principal do DSVS
\end{enumerate}

\section{Diagrama UML}

\section{Funcionamento do software}
\end{document}
