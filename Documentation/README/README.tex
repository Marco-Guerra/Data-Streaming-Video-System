%% Incluindo os pacotes necessários
\documentclass[12pt, a4paper]{article}
\usepackage[utf8]{inputenc}
\usepackage[portuguese]{babel}
\usepackage{titlesec}
\usepackage{titling}
\usepackage{enumitem}
\usepackage{indentfirst}
\usepackage{graphicx}
%%\graphicspath{{images/}}
\usepackage{fancyhdr}
\usepackage{color}
\usepackage{fancyhdr}
\usepackage{colortbl}
\usepackage{framed}

%% Definindo o Autor e o título
\author{Gustavo Juliano Borges \and Victor Emanuel Almeida \and Marco Aurélio Guerra Pedroso}
\title{Documentação do Projeto I~-~IES}

%% Estilo da página
\pagestyle{fancy}
\fancyhead[L]{Documentação~-~Projeto I}
\fancyhead[R]{Introdução Engenharia de Software}
\fancyfoot[L]{Outubro~-~2020}
\fancyfoot[C]{}
\fancyfoot[R]{-~\thepage~-}
\renewcommand{\headrulewidth}{0.7pt}
\renewcommand{\footrulewidth}{0.3pt}

%% Definindo espaçamento
\titlespacing{\section}{0pt}{*2}{*1.1}
%%\titlespacing{\subsection}{}{}{}

\titleformat{\section}
{\Large\bfseries}
{\thesection}
{.5cm}
{}

\pretitle{\vfill\begin{center}\LARGE}
\postdate{\end{center}\vfill}

\begin{document}
\begin{titlepage}
\maketitle\thispagestyle{empty}
\end{titlepage}
\section{Introdução}

Foi proposto pelo professor a elaboração de um programa que simula um streaming de vídeo.

Sendo assim criamos uma equipe formada por~\textbf{\theauthor}, e nomeamos nosso software de \textbf{Data Streaming Video System} \textbf{(DSVS)}

\section{Casos de uso}

\subsection{Cadastrar um vídeo}
Caso de Uso no qual o usuário final cadastra um novo vídeo a base de dados de vídeos.
\begin{enumerate}
	\item Ter q logar com seu user (Talvez)
	\item Entrar no menu Principal do DSVS
	\item Entrar no Menu de vídeos
	\item Entrar no menu de adicionar vídeos
	\item Entrar com os dados do vídeo
		\begin{enumerate}
			\item Entrar com ID (Talvez)
			\item Entrar com o tipo de vídeo
			\item Entrar com o nome do vídeo
			\item Entrar com o nome do diretor
			\item Entrar com a duração do vídeo
				\begin{itemize}
					\item quantidade total de horas
					\item quantidade total de minutos
					\item quantidade total de segundos
				\end{itemize}
			\item Entrar com o número de temporadas
			\item Entrar com o número de gêneros
			\item Entrar com os gêneros
		\end{enumerate}
	\item Voltar ao menu de vídeos
	\item Voltar ao menu principal
	\item Encerrar o programa
\end{enumerate}
\subsection{Cadastrar um novo usuário}
	Caso de Uso, no qual ocorre um erro no qual o usuário, já existente ou novo, cadastra-se.
\begin{enumerate}
	\item Ter q logar com seu user (Talvez)
	\item Entrar no Menu de usuário
	\item Entrar no Menu de adicionar usuário
	\item Entrar com os dados do usuário
		\begin{enumerate}
			\item Entrar com o ID (Talvez)
			\item Entrar com o nome
			\item Entrar com a data de nascimento (Ano $>$ Ano-Atual)
			\item FALHA
		\end{enumerate}
	\item Mensagem de falha
	\item Encerrar programa
	
\end{enumerate}

\section{Diagrama UML}

\section{Funcionamento do software}
\end{document}




% Menu principal
	%menu de video
		% aqui ainda recebe &
	%menu de user
